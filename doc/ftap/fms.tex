\documentclass[]{article}
\usepackage{lmodern}
\usepackage{amssymb,amsmath}
\usepackage{ifxetex,ifluatex}
\usepackage{fixltx2e} % provides \textsubscript
\ifnum 0\ifxetex 1\fi\ifluatex 1\fi=0 % if pdftex
  \usepackage[T1]{fontenc}
  \usepackage[utf8]{inputenc}
\else % if luatex or xelatex
  \ifxetex
    \usepackage{mathspec}
    \usepackage{xltxtra,xunicode}
  \else
    \usepackage{fontspec}
  \fi
  \defaultfontfeatures{Mapping=tex-text,Scale=MatchLowercase}
  \newcommand{\euro}{€}
\fi
% use upquote if available, for straight quotes in verbatim environments
\IfFileExists{upquote.sty}{\usepackage{upquote}}{}
% use microtype if available
\IfFileExists{microtype.sty}{\usepackage{microtype}}{}
\ifxetex
  \usepackage[setpagesize=false, % page size defined by xetex
              unicode=false, % unicode breaks when used with xetex
              xetex]{hyperref}
\else
  \usepackage[unicode=true]{hyperref}
\fi
\hypersetup{breaklinks=true,
            bookmarks=true,
            pdfauthor={},
            pdftitle={Finance, Mathematics, Software},
            colorlinks=true,
            citecolor=blue,
            urlcolor=blue,
            linkcolor=magenta,
            pdfborder={0 0 0}}
\urlstyle{same}  % don't use monospace font for urls
\setlength{\parindent}{0pt}
\setlength{\parskip}{6pt plus 2pt minus 1pt}
\setlength{\emergencystretch}{3em}  % prevent overfull lines
\setcounter{secnumdepth}{0}

\title{Finance, Mathematics, Software}
\author{true}

\begin{document}
\maketitle

This slim pamphlet describes a unified approach to pricing, hedging, and
evaluating the risk of derivative securities. The foreign exchange,
fixed income, equity, and commodity arenas have developed models that
get attached to a string of pioneers: Garman-Kohlhagen,
Black-Scholes/Merton, Heston, Black-Derman-Toy, Vasicek,
Heath-Jarrow-Morton, Gabillion. The simple fact is that they are all
stochastic processes with parameters that get tuned to the market and
are then used to come up with a putative price based on the cost of the
initial hedge. The current theory is not very good at evaluating the
practical risks involved with the theoretical hedge.

You may not understand the following statement yet: There is only one
mathematical model for derivative securities, a vector-valued stochastic
process; it is simply a matter of how you parameterize it. The goal is
to make this obvious and even useful.

\section{Finance}\label{finance}

Before learning about derivative securities, you need to know something
about the primary securities they are based on: who the market
participants are, what kinds of securities are traded, the markets
available for trading, and the motivations of the participants.

Every financial transaction is the exchange of some quantity of a
security for a quantity of another security at some point in time
between two counterparties. Usually one of the securities is the
currency of the country you live in. Owning a security involves cash
flows. In the foreign exchange market you get paid the difference
between the rates of interest in the currencies pairs you hold each day,
in fixed income you get paid coupons on a regular schedule, stocks pay
uncertain dividends on dividend dates. It is important to distinguish
the price of a security from the cash flows associated with owning it.

It is also useful to think of every security as a currency. This point
of view allows a unified approach to the problem of pricing, hedging,
and managing the risk of derivative securities. It might seem odd at
first to consider a stock to be a currency, but anything that can be
converted to or from a currency is fungible.

\subsection{Participants}\label{participants}

Every transaction has a buyer and a seller. Don't confuse these nouns
with the verbs buy and sell. The buyer always decides whether or not to
buy or sell based on the prices offered by a seller. A buyer can also
decide to go long or short. If a buyer goes long it means they own
something that they believe increases in value if the market goes up. If
a buyer goes short, they believe they make money if market goes down.
Prices are determined by the market. The value of a position is a
subjective judgement made by a buyer often based on a model.

You can already figure out what buy side/sell side means. These are also
referred to as retail/wholesale and price taker/maker.

Sellers make a market for buyers. They provide prices for buyers who
decide to buy at the ask, or sell at the bid. The ask price is also
called the offer, but the seller is not really offering anything other
than to allow the buyer to transact. The price also depends on the
amount being transacted and who the buyer and seller are. Markets are
never perfectly liquid, something the classical theory of mathematical
finance likes to assume. Sellers make their living off of this
asymmetry, among other things.

There are also exchanges and broker-dealers that facilitate trading.
They act as both buyers and sellers to facilitate transactions.
Exchanges provide their customers access to liquidity providers. The
exchange aggregates these quotes and show their customers the best
bid/ask spread currently available. Broker-dealers provide a similar
service but often hold an inventory of trades when they can't match a
buyer and a seller or want to trade their own account. Exchanges make
money proportional to the volume of trades whereas broker-dealers are
can take advantage of market movements instead of making money only on
the spread. Or take a hit on market movements. No exchange has every
gone bankrupt.

\subsection{Securities}\label{securities}

A security is the catch-all term we will use for any financial
instrument. The primary securities are currencies, bonds, stocks,
mortgages, and commodities. Stocks are called equity because they entail
partial ownership in a company. Owning stock gives you some say in how
the company is run. Bonds are called debt because the company issuing
the bond is on the hook to pay it back. If a company defaults, the bond
holders are first in line and the stock holders get what is left over.
Bonds are also called fixed-income instruments since they usually pay a
fixed coupon. This is not the only example of historical terminology
being a hindrance to a unified understanding of financial products.

\subsection{Cash Flows}\label{cash-flows}

Owning a security involves cash flows. Exchanging currencies involves
the difference between the interest rates (roll), bonds pay coupons and
principal, stocks pay dividends if long or may require carry if short.
Mortgages can't be shorted and pay interest and potentially early
principal. Commodities require storage costs.

Primary securities can be bundled, or repackaged into mutual funds,
mortgage-backed securities, exchange traded funds, special purpose
vehicles and an assortment of other securities. Often the purpose is to
take advantage of tax, accounting, or regulatory considerations.
Something we will assiduously avoid considering, as does most of the
mathematical finance literature, even though these are a very important
component of the financial world.

\subsection{Derivatives}\label{derivatives}

A derivative security is a contract between a buyer and a seller for a
sequence of exchanges of other securities. The most common are futures,
forwards, and options. The following chapters will show in detail how
these are the building blocks of all derivative securities. One example
are swaps. They are just a portfolio of forward contracts.

\subsection{Markets}\label{markets}

Immediate transactions are called spot (or cash) trades. These can be
exchange traded or over-the-counter.

Exchanges provide nearly instantaneous access to transactions via market
orders. The price cannot be guaranteed by the exchange. It depends on
the current order book and other market orders that might get placed
before yours.

The order book consists of limit orders. A limit order is a way for
buyers to act like sellers. You can offer to buy or sell some quantity
of an instrument at a price. If the market moves to the level you offer,
then you can acquire or relinquish the underlying at the price you set.

OTC transaction are done directly between two counterparties and involve
machinery more complicated than setting up a margin account with an
exchange. A single trade can take weeks or months of negotiation and
often involves setting up collateral agreements in place of margin
accounts. Exchanges provide a fixed menu of trades, the value added by
what used to be called investment banks are custom tailored trades. That
is the primary topic of this treatise.

Swap Execution Facilities (SEF) are similar to exchanges that trade
interest rate swaps. They were created to streamline OTC transactions
for vanilla swaps.

\end{document}
